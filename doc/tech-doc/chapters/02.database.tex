%!TEX root = ../Documentazione.tex
%%--------------------------------------------------------------------------
%% DATABASE
%%--------------------------------------------------------------------------



\chapter{Database}

\section{Struttura}
\label{structure}
Il database SQLite è stato implementato tramite SugarORM\footnote{\todo{Link}}, un ORM (Object-Relational Mapping) sviluppato appositamente per Android. In questo modo le entità del database sono direttamente mappate su classi Java, utilizzando il pattern DAO (Data Access Object).\\
Il database contiene 5 entità all'interno del package \\\code{it.gbresciani.legodigitalsonoro.model}:
\begin{description}
	\item[\code{Word}] Contiene la parola, sia in italiano sia in inglese, e le sillabe che lo compongono.
	\item[\code{Syllable}] Contiene la sillabla e l'esadecimale del colore associato.
	\item[\code{WordStat}] Permette di tracciare il momento in cui una parola viene trovata.
	\item[\code{GameStat}] Contiene le statistiche di una particolare partita
\end{description}
In figura \ref{fig:entities} sono riportati dettagli delle entità.

\begin{figure}[h!]
\label{fig:entities}
  \centering
    \centering{
      \includegraphics[width=\textwidth]{entities.jpg}}
  \caption{}
\end{figure}

\section{Inizializzazione}
Il database viene inizializato la prima volta che l'applicazione viene aperta con i dati in formato \textit{JSON} presenti nei files \code{words.json} e \code{syllable.json} contenuti nella cartella degli \code{assets}.\\
Il listing \ref{lst:words_json} contiene un estratto del file \code{words.json}

\begin{lstlisting}[float, caption=Porzione del file \code{words.json}, label=lst:words_json]
[
  {
    "lemma":"arco",
    "eng":"arc",
    "syllable1":"ar",
    "syllable2":"co"
  },
  ...
\end{lstlisting}

Il mapping degli elementi JSON con la classe Java associata avviene tramite la libreria Open Source GSON\footnote{\todo{Link}}, sviluppata da Goolge, che fa uso di annotazioni per mappare gli attributi delle classi con i campi JSON.\\
L'operazione di conversione dei dati da JSON a classi Java e il successivo inserimento all'interno del database, in quanto operazioni relativamente lunghe, non vengono eseguite all'interno del \textit{Thread} principale ma in uno differente grazie ad un \code{IntentService} contenuto in \\\code{it.gbresciani.legodigitalsonoro.services.InitDBService}. 

