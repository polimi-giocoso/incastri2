%!TEX root = ../Documentazione.tex
%%--------------------------------------------------------------------------
%% LOGICA DI GIOCO
%%--------------------------------------------------------------------------



\chapter{Logica di gioco}

\label{chap:game_logic}

La logica di gioco è affidata all'Activity \code{PlayActivity}; essa gestisce inoltre i \code{Fragments} (vedi capitolo \ref{chap:ui}) che visualizzano l'Intefaccia Utente.

\section{Glossario}
\label{sec:glossary}
In questa sezione vengono descritti acuni termini utili alla comprensione delle successive sezioni.

\begin{description}
\item[Page] Una partita è composta da una o più pagine. Una pagina può contenere 2 o 4 sillable e da 1 a 4 parole da trovare
\item[User role] Durante una partita multi player i giocatori si distinguono in ``master'', colui che ``orchestra'' la partita, e ``slave'', colui che vi partecipa in modo passivo 
\item[Turno] Durante una partita multi player rappresenta la possibilità per un giocatore di provare a indovinare una parola. Il giocatore che non è in possesso del turno attende la mossa dell'altro
\end{description}

\section{Event Bus}
\label{sec:event_bus}
Per far comunicare i vari componenti dell'applicazione (\code{Activities}, \code{Fragments}, \code{Services}) è stato utilizzato Otto, un \textit{Event Bus}, sviluppaato appositamente per Android, che permette ai vari componenti di postare eventi su un unico Bus e per riceverli in modo asincrono.\\
La classe \code{BusProvider} permette, tramite un metodo statico, di accedere allo stesso Bus in ogni punto dell'applicazione.\\
All'interno del package \code{package it.gbresciani.legodigitalsonoro.events} son presenti numerose classi (ad esempio \code{WordClickedEvent} o \code{PageCompletedEvent}) che rappresentano i diversi eventi generati e ``ascoltati'' dai componenti dell'app.\\
La classe \code{NineBus} è una piccola estensione del Bus predefinito di Otto che permette di pubblicare tutti gli eventi sul \code{Main Thread}, anche se generati in un altro \code{Thread}.

\section{GameState}
\label{sec:game_state}
\code{GameState} è la classe che rappresenta lo stato di una partita, sia essa single o multi player. Viene inizializzato da \code{PlayActivity}, e aggiornato durante il corso della partita in seguito ai diversi eventi prodotti dagli altri componenti dell'app.\\
I suoi attributi sono:
\begin{description}
    \item[pageNumber] Il numero di pagina corrente.
    \item[pages] Il numero di pagine totali della partita in corso (stabilito nelle impostazioni).
    \item[pageWordsToFindNum] Il numero di parole ancora da trovare nella pagina corrente.
    \item[wordsAvailable] Un \code{ArrayList} contentente le parole ancora da trovare.
    \item[wordsFound] Un \code{ArrayList} contenente le parole gia trovate.
    \item[syllables] Un \code{ArrayList} contenente le syllabe presenti nella pagina corrente.
    \item[currentPlayer] Se la partita è multi player contiene una stringa che rappresenta il ruolo del giocatore che detiene in quel momento il turno.
    \item[currentPlayerDeviceId] Se la partita è multi player contiene una stringa che rappresenta l'id del dispositivo che detiene in quel momento il turno.
\end{description}

La classe \code{GameState} contiene attributi annotati con \code{@SerializedName} in modo da poter essere convertita in un oggetto JSON (tramite GSON) ed essere inviata al secondo giocatore nel caso di partita multiplayer (maggiori dettagli nella sezione \ref{sec:multiplayer}).

\section{Statistiche di gioco}
\label{sec:stats}
Le satistiche della partita, come accennato nel capitolo sul database, vengono salvate tramite i DAO \code{WordStat} e \code{GameStat} che contengono, ad esempio, i tempi di ritrovamento delle parole, le parole stesse, la pagine su cui sono state trovate o eventualmente l'id del giocatore che le ha trovate.

\section{Game flow}
\label{sec:game_flow}
Tutta le fasi di una partita sono gestite dall'\code{Activity} \code{PlayActivity}.
All'avvio, cioè durante la chiamata al metodo \code{onCreate()} essa esegue le seguenti operazioni:
\begin{enumerate}
\item Carica i settaggi della partita
\item Carica i suoni di gioco
\item Istanzia il sintetizzatore vocale \code{TextToSpeech}
\item In base al tipo di partita (single o multi player) inizializza o meno il Bluetooth e setta la variabile \code{multi} a ``MASTER'' o ``SLAVE''
\end{enumerate}

I principali metodi di \code{PlayActivity} sono:
\begin{description}
\item[\code{startGame()}] Si occupa di inizializzare gli oggetti per le statistiche e chiama \code{constructPage()} and \code{startPage()}
\item[\code{constructPage()}] Inizializza lo stato della partita (\code{GameState}) con i dati relativi alla pagina corrente. Chiama i metodi accessori \code{Helper.chooseSyllables()} (sceglie in modo casuale le sillabe per la pagina) e \code{Helper.permuteSyllablesInWords()} (calcola le permutazione delle sillabe scelte per trovare le parole componibili)
\item[\code{startPage()}] Inizializza i \code{Fragments} che mostreranno l'interfaccia utente
\item[\code{updateState()}] Si occupa di gestire la logica di gioco: determina se tutte le parole sono state trovate o se interrompere il gioco quando non è il proprio turno e posta sul Bus l'evento \code{StateUpdatedEvent} in modo che i \code{Fragments} possano reagire correttamente
\end{description}

L'\code{Activity} presenta inoltre diversi metodi annotati con \code{@Subscribe} che permettono di reagire agli eventi pubblicati da altri componenti.

\section{Invio statistiche}
\label{sec:stats_send}
L'invio delle statistiche avviene tramite l'invio di una mail all'indirizzo settato nelle impostazioni dell'applicazione.\\
L'invio è gestito dall'\code{IntentService} \code{GenericIntentService} al quale viene passato l'id corrispondente alla partita appena conclusa. Interrogando il database esso carica i dati e li inserisce nel corpo dell'email formattandoli come mostrato nel listing \ref{lst:stats}. Il listing \ref{lst:stats_example} ne mostra invece un esempio.

\begin{lstlisting}[float, caption=Struttura delle satistiche inviate, label=lst:stats]
Dispositivo N: device id
Tempo totale: mm:ss
N - word: mm:ss (# of page on which the word was found)
...
\end{lstlisting}


\begin{lstlisting}[float, caption=Esempio di statistiche, label=lst:stats_example]
- One player -

Dispositivo 1: BC:20:A4:73:6B:49
Tempo totale: 00:13
1 - fumo: 00:02 (1)
1 - muro: 00:13 (2)

- Two Player -

Dispositivo 1: BC:20:A4:73:6B:49
Dispositivo 2: D4:0B:1A:15:FD:AD
Tempo totale: 00:14
2 - ceno: 00:02 (1)
1 - noce: 00:09 (1)
2 - buco: 00:14 (2)
\end{lstlisting}

\section{Multiplayer}
\label{sec:multiplayer}
Come già detto la modalità multi player segue le stesse regole di quella single player se non per il fatto che i due giocatori provano a turno a trovare una parola.\\
Il tablet ``master'' è quello che inizia il processo di connessione scegliendo, tra i dispositivi in ascolto, il tablet ``slave''. I vari aspetti della partita come i settaggi, le sillabe presenti e le parole da trovare vengono tutti stabiliti dal ``master'' che si occupa quindi anche di gestire il flow del gioco, mentre lo ``slave'' partecipa in modo passivo.\\
La comunicazione tra i due dispositivi avviene tramite Bluetooth. La classe che si occupa di stabilire la connessione è \code{BluetoothService}; essa si serve di tre \code{Thread} differenti (\code{AcceptThread}, \code{ConnectThread}, \code{ConnectedThread}) per ricevere un connessione, effettuare una connessione, e inviare e ricevere messaggi.\\
I messaggi scambiati sono stringhe contenenti oggetti JSON preceduti da un header che ne identifica la natura. I possibili messaggi sono $4$:

\begin{description}
    \item[GAME\_STATE] Contiene la rappresentazione JSON dell'oggetto \code{GameState} descritto nella sezione \ref{sec:game_state}. Solo il ``master'' invia questo tipo di messaggio allo ``slave'', il quale ne ricava tutte le informazioni necessarie.
    \item[SIMPLE\_TURN\_PASS] Lo ``slave'' invia questo messaggio al ``master'' quando non è riuscito a trovare parole e quindi sta semplicemente passando il turno.
    \item[WORD\_FOUND] Lo ``slave'' invia questo messaggio insieme alla parola trovata, in modo che il ``master'' possa aggiornare lo stato della partita.
    \item[GAME\_END] Il ``master'' invia questo messaggio allo ``slave'' al termine della partita.
\end{description}